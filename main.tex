\documentclass{article}
\usepackage[utf8]{inputenc}
\usepackage{todonotes}

\title{rapport}
\author{jjycavailles }
\date{May 2019}

\usepackage{natbib}
\usepackage{graphicx}


%\usepackage{hyperref}
\usepackage[colorlinks=true, allcolors=blue]{hyperref}

%\bibliography{references.bib}
%\bibliography{intro.bib}
%\addbibresource{intro.bib}


\begin{document}

\begin{titlepage}

\begin{center}
  \includegraphics[width = 25mm]{LogoInsa.png} \hfill
  \includegraphics[width = 30mm]{logo_cnrs.jpg}
\end{center}

%\title{\textbf{Rapport de stage} \\ Ingenieur en mathematique \\ \textbf{ etude de la classification en regimes de temps et de leurs impacts pour des utilisations metiers.} }
%\author{Jerome Cavailles \\ $4^{ieme}$ annee, Genie mathematique et modelisation \\ INSA Toulouse}
%\date{28/06/2018 - 07/09/2018}




\vspace*{1cm}

\begin{center}
\rule{\linewidth}{0.7mm} \\
[0.4cm]
\textbf{ \Huge Internship report} \\
[0.2cm]
\large \emph{Engineer in mathematics} \\ 
[0.6cm]
\textbf{ \huge Destabilizing effects of controlling ecosystem behavior} \\
[0.4cm]
03/01/2019 - 07/31/2019 \\
[0.4cm]
\rule{\linewidth}{0.7mm}
\end{center}

\vspace*{0.5cm}

\begin{center}
\textbf{\Large{Jerome Cavailles}} \footnote{\url{jcavaill@etud.insa-toulouse.fr}} \\ [0.3cm] $5$ years, Mathematical engineering and modeling \\ INSA Toulouse
\end{center}

%^{\text{ieme}}

%\maketitle

\vspace*{3cm}

\begin{flushleft}
\hfill 
Supervisor : Yuval Zelnik \footnote{\url{yuval.zelnik@sete.cnrs.fr}} \\
Tutor : Robin Bouclier \footnote{\url{bouclier@insa-toulouse.fr}, \url{jean-yves.dauxois@math.univ-toulouse.fr}}  \hfill 
Michel Loreau \footnote{\url{michel.loreau@sete.cnrs.fr}}
University : INSA Toulouse\footnote{135, Avenue de Rangueil 31077 Toulouse Cedex 4} \hfill
Laboratory : CNRS-Moulis\footnote{2, route du CNRS - 09200 Moulis, France, \url{http://www.cbtm-moulis.com}} \\
\end{flushleft}

\end{titlepage}



\newpage
%\addto\captionsfrench{\def\contentsname{}} % pour supprimer le "table des matieres en haut"

\paragraph{}
%\section*{Contents}

\addcontentsline{toc}{section}{Contents}

\tableofcontents



\newpage
%\section*{Liste des figures, des tables et des algorithmes}
%\addcontentsline{toc}{section}{Liste des figures, des tables et des algorithmes}
%\paragraph{}
\addcontentsline{toc}{section}{List of figures}
\listoffigures



\newpage
\addcontentsline{toc}{section}{Abbreviations}
\todo{use nomenclature packages}
\paragraph{Abbreviations}
%\listoffigures





\newpage
\section*{Acknowledgment}
\addcontentsline{toc}{section}{Acknowledgment}



\newpage
\todo{Choice of this internship ?}
\section*{Introduction}
\addcontentsline{toc}{section}{Introduction}

\subsection*{Station presentation}
\addcontentsline{toc}{subsection}{Station presentation (1-2 pages)}

\paragraph{}
The CNRS\footnote{\url{http://www.cnrs.fr/en/cnrs}}, the Scientific Research National Center (in french, Centre National de la Recherche Scientifique) was created on the 19th of October, 1939. It is a world renowned research institution, ranked second by  nature index \footnote{\url{https://www.natureindex.com/institution-outputs/generate/All/global/All/n_article}}. It has approximately 33,000 researchers working in 1,144 laboratories throughout France and abroad, with a budget around 3 billion euros. 

The CNRS is currently headed by Antoine Petit (President and CEO), and its laboratories are organised in two types: proper units (UPRs) and mixed units (UMRs), the latter being managed in association with other French institutions (higher education establishment or another research institution). In addition, there are 36 international Joint Units (UMI) of collaborations around the world. The CNRS conducts research in all disciplines of basic research (Ecology and environment, Humanities and social sciences, Engineering and systems, Mathematics, Physics, Information sciences, etc. ).


\paragraph{}
One of these mixed research units (UMR 5321) is the Station for Theoretical and Experimental Ecology\footnote{\url{http://www.ecoex-moulis.cnrs.fr/}} (SETE), located in Moulis\footnote{\url{http://www.communes.com/midi-pyrenees/ariege/moulis_09200/}} (Ariege, France). It was originally founded in 1948 by professors Jeannel and Vandel, due to its vicinity to many caves, with the aim of the station to use the underground cave systems in order to study the formation and physical properties of karstic systems as well as systematics and adaptations in hypogeaic organisms. More recently, under the direction of Jean Clobert\footnote{\url{http://www.ecoex-moulis.cnrs.fr/spip.php?article26}}, the station transitioned to perform more general research about ecology.

The research station is now directed by Michel Loreau\footnote{\url{http://www.ecoex-moulis.cnrs.fr/spip.php?article47}}, and has a staff of 60 persons working in it, along several axes : biodiversity and ecosystem functioning, conservation and land management, social interactions, environmental variation and evolution, phenotypic plasticity. 
%% YZ: It would be good to define the axes more properly and conciesly
Several unique experimental platforms are located in the station. A laboratory inside the cave system is still operational, a $750m^2$ greenhouses have been constructed, a $520m^2$ aviary with an automatic system for data capture using video and sensors.
The station also has equipment for molecular biology, cell biology, physiology, for surgery and also to breeding of invertebrates, fish, amphibians, and reptiles. However, the most unique facility is the metatron\footnote{\url{https://themetatron.weebly.com/}}, which is a network cell semi-controlled. 
%% YZ: This phrasing "network cell semi-controlled" is strange. You might want to mention size and such things, to give the reader an idea of what it is. You can even add a photo here, perhaps.
Each unit reproduce a small ecosystem (with vegetation and insects). Cells can be linked to study the dispersal of the species. By controlling the temperature, it is possible to investigate the consequences of the global change.

%% YZ: There's also the aqua-tron, that has recently been finished. You can talk to several people in Jose's team about it, if you're interested.


\paragraph{}
Since 2012, the station hosts the Centre for Biodiversity Theory and Modelling (CBTM) \todo{now renamed Linking ...} 
%% YZ: Actually, not quite. The CBTM should still exist as a separate entity, in parallel to the "linking team". I think for your concerns, it might be simpler to just describe the CBTM...
which aims to unify theories of biodiversity changes and of their consequences. 
%% YZ: I think you can say more than the previous sentence...
Lead by Jose M. Montoya\footnote{\url{http://www.cbtm-moulis.com/m-224-jose-m--montoya.html}}, the research ranges from phylogenetics to human interactions. Indeed, the ambition of the team is to give a theoretical framework to a general biodiversity science in order to deal with the present biodiversity crisis\footnote{\url{https://www.ipbes.net/news/Media-Release-Global-Assessment-Fr}}.

In practice, the team focus on several axes : 
%% YZ: I guess you should state the axes right here. Otherwise, it is quite hard to follow.
 First, trying to understand how biodiversity change will affect ecosystem services. For example, how the loss of a species can affect crop production. However, because ecology and evolution theory are interrelated, the team integrates the role of eco-evolutionary dynamics in the responses of environmental changes. The same work is done for structure, dynamics and function of ecosystem, which are also interconnected \cite{bastazini2019loss}\cite{bideault2019temperature}\cite{galiana2019geographical}.

An additional axis is of human-nature interactions, which is studied in both directions : the human impact on biodiversity (habitat loss, fragmentation, global warming, etc.) which represent a threat of at least one in six species during this century, but also the feedback of biodiversity loss on human society. The aim is to study the long term sustainability of coupled social-ecological systems. Another objective is to better understand how biodiversity changes affect the services of the ecosystem, in particular  crop production and biological control in agricultural landscapes
\cite{cazalis2018we}\cite{lafuite2018sustainable}\cite{montoya2018trade}\cite{montoya2019trade}.

% Habitat fragmentation and spatial dynamics of biodiversity
A particular case of human impact, landscape fragmentation which occurs at multiple spatial scales, is itself a subject of research: the habits loss itself as well as its effect ton he spatial dynamics of the ecosystem, with a special attention to metacommunities dynamics \cite{goncalves2018habitat}\cite{jacob2018habitat}.

% Biodiversity and stability of ecological systems
Finally, a major focus of the team is studying the stability of ecological systems\footnote{\url{http://www.cbtm-moulis.com/m-214-biostases.html}}. Stability is an important feature of the ecosystem, and is a notion that is used in all previous research axes mentioned. The main research focus is to understand and quantify stability both in time and in space \cite{wang2017invariability}\cite{zelnik2018impact}. However, since different stability measures are used in theory and experimental ecology, the team has tried to bridge the gap between these different measures and thus unify the notion of the stability \cite{arnoldi2018ecosystems}\cite{barbier2019pyramids}.
%% YZ: I am not sure barbier2019pyramids makes sense here. What measures is it bridging?
Different aspect of stability are studied: the link between the diversity of a species community and the stability of the community\cite{vallina2017phytoplankton}; the stability of meta-ecosystems (food webs, synergies between multiple ecosystem)\cite{arnoldi2016particularity}\cite{lurgi2016effects}\cite{wang2016biodiversity}. 
%% YZ: I guess "food webs, synergies between multiple ecosystem" is meant to explain what meta-ecosystems are. I think this is not a good explanation. Maybe talk about metacommunities, and look at Leibiold2004 paper to see what is a good definition
Again, the sustainability of coupled social?ecological systems are studied, in as stability point of view (averting collapse for example).
%% YZ: This last sentence is quite strange.
 Moreover, a mathematical framework is built by exploring the different notions of stability in order to link them \cite{arnoldi2018ecosystems}\cite{arnoldi2016unifying}\cite{donohue2016navigating} and to know how to predict a critical changes by using experimental measures such as temporal variability\cite{arnoldi2016resilience}\cite{haegeman2016resilience}\cite{wang2017invariability}. 
% citation fouille jusqu'a 2016 inclus



\newpage


\subsection*{Context}
\addcontentsline{toc}{subsection}{Context}

\subsubsection*{Ecosystem management}
\addcontentsline{toc}{subsubsection}{Ecosystem management}


%\paragraph{changed actual \\}
\paragraph{} %\todo{reformulate}

% Nature change
%One of the primary challenges of our time is to enhance global food production and security.
%\item We understand sustainable development as the Brundtland Commission defines it as ?development that meets the needs of the present without compromising the ability of future generations to meet their own needs? \cite{brundtland1985world}
It is common knowledge that ecology actually faced one of the more important crisis \cite{oosthoek_humanity_2005}.
%% YZ: This is very strange phrasing. I would think to use something like "Ecosystems around the world are facing unprecedented disturbances ... due to increasing human intervention"
It is therefore important to understand the dynamics the natural environment in order to both predicts the following state of the ecosystem and also to better manage it.

% Complex dynamics
Ecosystem dynamics can be notoriously complex, in particular due to the interactions between the different interconnected elements that constitute the system. Thus, it is impracticable to use traditional cartesianism decomposition to study the mechanics of the nature.
%% YZ: I don't really know what you mean by "traditional cartesianism decomposition"
Moreover, ecologists are often interested in estimating stability far from equilibrium (as opposed to the classical physics approach, which focuses on studying stability near the equilibrium). % need news tools to understand it
Therefore, ecology needs to develop new approaches to deal with understanding and predicting stability in complex dynamical systems. 
% By using appropriate tools to tackle 
% One of this is to study the "stability" of ecological dynamics. It is relevant to anticipate the consequences and the viability of ecosystem due to perturbation (from human and/or natural).



\paragraph{} % measures of the stability, resilience
%\todo{expend or do appendices to talk more about this (stability)}
% POLICYMAKERS and ECOLOGISTS needs measures 
One of the needs for both policy makers and ecologists is to use well defined measures to quantifies ecosystem stability. 
%% YZ: This last sentence probability should be rephrased, since you talk about policy makes and ecologists with no previous context.
Such measures aim to quantify the health of an ecosystem and follow its development over time, and are used to set accurate goals for the future planning and management \cite{donohue_navigating_2016} \cite{mayer2008strengths}. 

% 163 definitions of 70 different stability concepts
It is important to clarify the different notions related to the concept of stability. The main difficulty is that this word is used for many different meanings. One study has identified 163 definitions of 70 different stability concepts \cite{grimm1997babel}.% 6 main Stability still according to  \cite{grimm_babel_1997}
But, according to this study, all these can be collapsed to only 6 pertinent concepts (constancy, resilience, persistence, resistance, elasticity and domain of attraction). Indeed, these notions are interrelated and can thus be reduced in some contexts \cite{donohue2013dimensionality}.

One such concept, resilience, is traditionally used in theoretical studies, but is not the most relevant: new measure need to be introduced to bridge the gap between experimental and theoretical research \cite{arnoldi2016resilience} \cite{gunderson2000ecological} \cite{neubert_alternatives_1997}.
%% YZ: I am not sure where you're going with this last sentence 
Moreover, even if it is possible to reduce the number of the stability notions, several of them need to be used in order to consider the various aspects of stability, so as not to lose information on the behavior of the ecosystem \cite{derissen_relationship_2011}. There is thus a compromise between using too few measures, which will not capture all the relevant information, and using too many measures, which will not be practicable and may still not catch all the information \cite{hillebrand2018decomposing}.


%%%%%% not relevant, I think

% \item the magnitude of disturbance that can be absorbed before the system changes its structure by changing the variables and processes that control behaviour \cite{holling2002resilience}

% \item Moderate disturbance?may include severe mortality events such as catastrophic hurricane damage, provided that there is sufficient time between recurrent events to allow for recovery (Rogers 1993)



%\paragraph{ecosystem management \\}
\paragraph{}

These different concept a main interest for ecosystem management \cite{mumby2014ecological}. It serves to anticipate the consequences of disturbances, in particular for anthropocentric ones. The impact of human activities on the natural environment can be both deliberate (when people choose to influence the ecosystem) or indirect (when the disturbances is a secondary effect of another goal). 
%% YZ: This distinction between deliberate and indirect does not make sense.
In the past decades, the field of ecosystem management has grown rapidly \cite{grumbine1997reflections} in response to the various modern disturbances, in order to sustain the integrity of ecosystem (including its structure, composition and function) \cite{jensen1994overview}. 

A major obstacle is to define measurable goal in order to have clear and trackable objective \cite{slocombe1998defining}.
%% YZ: how is goal and objective different?
Even if it remains impossible to know all the exact processes operating within the ecosystem, it is still possible to understand the dominant behavior, which could be sufficient for ecosystem management \cite{mori2011ecosystem}\cite{slocombe1998defining}\cite{stanley1995ecosystem}. 

%\item A goal usually has a wide, almost ethical, dimension of rightness. Objectives are the specific, doable tasks needed to achieve the goal \cite{slocombe1998defining}
%\item Sociological, ecological, technological, and economic information must be integrated \cite{jensen1994overview}

%\paragraph{too complex : Case study, forest fire management \\}
%\paragraph{Forest fire \\}
\subsubsection*{Forest fire}
\addcontentsline{toc}{subsubsection}{Forest fire}

To make easier to understand, a case studied is choose.
%% YZ: You need to say more about why we do a case study. 
We focus on forest fire management, as the dynamics of both forest and fire are well established. Nonetheless, the repercussions of fire management in forests, are not well understood, with much to be explored.% (in other management in connection to both forest dynamics and fire control) stay interesting.
%% YZ: I didn't understand " (in other management in connection to both forest dynamics and fire control)", so I took it out for now



\paragraph{}
%\paragraph{forest disturbance \\}

Forest dynamics are affected by various disturbances. We define disturbance as events that can cause significant changes to the ecosystem \cite{white1985natural} \cite{rykiel1985towards}. Perturbations play an important role in forest ecosystem, and could even be a driver of evolution (notably by generating heterogeneity in the landscape) \cite{turner2010disturbance}.
%% YZ: perturbations or disturbances? you should probably choose which one to use, or define both separately...
%% YZ: Also, not sure why you bring in evolution. It seems like a divergence...

Disturbances can be defined by their duration \cite{perera2015simulation}. First, the perturbations who are considered instantaneous comparative to the dynamics of the forest 
%% YZ: again, perturbations or disturbances? you should probably choose which one to use, or define both separately...
(e.g. earthquake, lava flow, landslide, flood, windstorm, ice storm, wildfire, pest outbreaks, clearing of land, flooding by beavers).
%% YZ: This list seems very long. Is that necessary?
Second, the constrain on long term, such as drought, water table fluctuation, temperature fluctuation, soil freeze, thaw cycles, soil erosion and deposition, disease, low-intensity harvesting, grazing.
%% YZ: Same for this list, also very long.

Another useful distinction is the implication of human in the disturbance, even if it is not possible to isolate anthropogenic perturbations from and natural ones \cite{perera2015simulation}.
Indeed, human impact has become increasingly important.
%% YZ: This last sentence seems to be very similar to an earlier one. Maybe more specific to forests?
For example, the area logged per year in Canadian forests has doubled between 1960 and 1995 \cite{smith_canadas_2000}. This logging disturbance could be relatively different from the natural ones.

Also, we some particular perturbations can be differentiate, the severe but rare events, this are termed "LIDS" for large and infrequent disturbances \cite{foster1998landscape}.
%% YZ: I don't follow this sentence 

Finally, this different perturbations are often interrelated \cite{keane2015exploring}. This could create synergism between them \cite{mandre_environmental_2011} or have unanticipated responses \cite{perera2015simulation}. For example, fire and climatic fluctuations could interact to product cumulative effects \cite{romme2009historical}.



%\paragraph{Forest management \\}
\paragraph{}
% From the earliest times, thoughtful people have encouraged the wise use of forests \cite{macdicken2015global}
For decades, sustainable forest management (SFM) has been used to maintain forest ecosystems \cite{macdicken2015global}. 
This practice serves to maintain different aspect of the ecosystem: forest resources, forest ecosystem health and vitality, productive functions, biological diversity, and socioeconomic functions \cite{makela_using_2012}. 
%% YZ: if you could shorten this list it would be good.
However, its main target is to conserve the forest ecosystem as an unified entity \cite{franklin1989toward}.
Moreover, there is no unanimity on this different facet of sustainability \cite{martinez-vega_assessing_2016}.
%% YZ: what facet? I don't understand
 Also, the human demands from forests have broadened, making forest management more complex \cite{eggers2017balancing}. Ecosystem management should try to reproduce the nature disturbance \cite{bengston_changing_1994} \cite{bengtsson2000biodiversity} in order to preserve the dynamics of the forest. 
 %% YZ: I guess this last line should start with "several recent studies have shown that..." or something like this?
 According to \cite{hunter1990wildlife} \cite{hunter1988paleoecology} it could be possible to imitate the size, frequency and severity
 %% YZ: Also this sentence needs to be reworked

To reach this different target, various criteria are used. 
%% YZ: what target is this?
To be practical, such criteria need follow some rules: be easily measured, be sensitive to stress, be anticipatory (to counter change), be integrative (consider different facets of forest ecosystem such soils, vegetation types ...) and have a low variability in response \cite{dale2001challenges}. However, monitoring programs typically consider only few indicators and fail to take into account the complexity of the ecosystem \cite{dale2001challenges}.


%\paragraph{Fire \\}
\paragraph{}

One on the main disturbances in forests is fire, and in some regions, it is the most significant one. Fires can create spatial patterns and heterogeneity in the landscape \cite{skinner1996fire}. Fires also affect plant behavior, such that plants develop traits for adaptation to fire (thick  bark  and fire-stimulated flowering, sprouting, seed release and/or germination) \cite{mckelvey1996overview} \cite{chang1996ecosystem}.

Fires are also linked with other perturbations, mostly climatic variation \cite{mckenzie_climatic_2004}\cite{da2018dynamics}, by its effects to on fuel \cite{schoennagel_interaction_2004} and by weather \cite{fernandes_fire-smart_2013}.
Moreover, fires are affected by humans. In some regions, wildfires have been considerably reduced due to the intervention of firefighters. In order to restore the natural dynamics of the forest, fires are sometimes allowed to run their course freely without intervention \cite{wallenius2011major}. 

Finally, one of the problem of the study of fire is its stochastic aspect and the variability in both space and time, which add complexity to the modelling of fire events and dynamics \cite{agee1998landscape} \cite{lertzman1998three}.





%\subsubsection*{ecosystem (general point of view)}
%\addcontentsline{toc}{subsubsection}{ecosystem (general point of view)}
%\begin{itemize}
%    \item Human have an important impact on natural ecosystem
%    \item ecosystem management (issue ... )
%    \item stability / resilience / robustness measures ? \todo{here, or in methods ?}
%    \item system too complex
%\end{itemize}
%\subsubsection*{forest fire management}
%\addcontentsline{toc}{subsubsection}{forest fire management}
%\begin{itemize}
%    \item Special case : forest fire (simpler study)
%    \item forest disturbances
%    \item fire control
%    \item fuel removal
%    \item fire control without fuel removal $\Rightarrow$ higher risk of collapse
%\end{itemize}




\subsection*{Synopsis}
\addcontentsline{toc}{subsection}{Synopsis}

\paragraph{\\}

The purpose of the present document is to demonstrate the risk for a collapse of a forest ecosystem due to the attemots to stabilize the ecosystem. More precisely, we explore the impact of managing forest fires; for example, suppression of wildfire can change the dynamics of the ecosystem and lead to a collapse.

%In order to study it, a numerical model of ordinary differential equations have been implemented.
The ecosystem is modeled with two coupled differentials equations, with a stochastic term in each equation, to model the effect of fire. This model have been studied mostly numerically.

First, the model is presented in detail, and an estimation of each model parameter is given according to the relevant ecology literature. In order to simplify the model, the a non-dimensionalisation is presented. Second, some basic model properties are specified and demonstrated (e.g. the equilibrium of the model). Furthermore, different kind of evolution of the model over time are presented. Finally, the impact of the fire frequency on the collapse probability are described. In particular, the link between the frequency of the fire and the risk of extinction is described. One purpose of this analysis is to determine when it is possible to detect a collapse risk by using other measures.




%%%%%%%%%%%%%%%%%%%%%%%%%%%%%%%%%%%%%%%%%%%%%%%%%%%%%%%%%%%%%%%%%%%%%%%%%%%%%%%%%%%%%%%%%%%%%%%%%%%%%%%%
% Methods
%%%%%%%%%%%%%%%%%%%%%%%%%%%%%%%%%%%%%%%%%%%%%%%%%%%%%%%%%%%%%%%%%%%%%%%%%%%%%%%%%%%%%%%%%%%%%%%%%%%%%%%%

\newpage
\section{Methods}


\subsection{Model}

\subsubsection{Model (original)}

\paragraph{Model of the forest}

\begin{itemize}
    \item write the model \todo{Rewrite the model remove of positive perturbation we need to see the minus n the equations) and put the parameter on $N$ (because $W$ control $N$}
    \item explain the coefficient
    \item write assumptions
\end{itemize}

\paragraph{Model of the fire}

\begin{itemize}
    \item write the model
    \item explain the coefficient
    \item write assumptions
\end{itemize}


\subsubsection{Adimensionnalisation}


link to calculus in appendix    





\subsection{Parametrization}

link between the forest literature and the model

\paragraph{Estimation of the coefficient of the original model}

\begin{itemize}
    \item forest model
    \begin{itemize}
        \item $g$
        \item $K$
        \item $A$
        \item $\mu$
        \item $d$
    \end{itemize}
    \item fire model
    \begin{itemize}
        \item $frequency$
        \item $s$
        \item $\alpha$
        \item $\beta$
    \end{itemize}
\end{itemize}



\paragraph{Estimation of the coefficient of the original model}

\begin{itemize}
    \item forest model
    \begin{itemize}
        \item $a$ 
        \item $m$ \todo{the ratio m/d give the stable equilibrium of $w$ and so the ratio w on the total biomass}
        \item $d$
    \end{itemize}
    \item fire model
    \begin{itemize}
        \item $frequency$ 
        \item $s$
        \item $\alpha$
        \item $\beta$
    \end{itemize}
\end{itemize}



\subsection{theoretical calculus}

just give few point to the dynamics of the model, the idea is not to give result but to help understand the dynamics of the model.

\begin{itemize}
    \item equilibrium
    \item stability
    \item time to come back to equilibrium 
    \item ...
\end{itemize}


\paragraph{different scenarios} \todo{exhaustive list, or not ?}
\begin{itemize}
    \item always come back to equilibrium
    \item always collapse (collapse really fast) system not viable
    \item fuel management (with enough frequency, $W$ remain low)
    \item ...
\end{itemize}



\subsection{Measures}

\todo{Variability analysis should be performed on data that are free from artefact \cite{seely2004complex} }

\todo{More time means more variation \cite{lawton1988more} }

\todo{A good measure of variability will be independent of the mean abundance if the dynamics are the same, but will not be independent if the dynamics change with mean abundance \cite{noauthor_temporal_1994}}

\todo{We are wary of Leps' statement that 'comparison of results based on various variability measures helps to draw the most unbiased conclusions'. One consequence of the realisation that different measures of variability are related to the mean (and to other factors) in different ways has been a tendency toward the calculation of multiple measures for individual time-series \cite{gaston_measurement_1993}}

\todo{Some authors (MacArthur 1972; Diamond 1984; Pimm et al. 1988) have argued that the rate  of extinction should be directly related to population variability. Al- though intuitively appealing, this relationship may not hold, especially when population density and absolute population variability are positively correlated (Tracy and George 1992; Schoener and Spiller 1992) \cite{rutledge1976ecological}}
\todo{ (N),Population size of mature individuals (and trendin population size, were clearly the best predictors of extinction risk) \cite{ogrady_what_2004}}




\todo{BELOW (list): \cite{grimm_babel_1997}, pas sur de la ref}
    \begin{itemize}
        \item Population models generally predict increased extinction risk (controversial results) 
        \item tests of the predicted effect of population variability (PV) have yielded variable, controversial results. Several studies provide apparent support for the predicted positive relationship (Karr 1982;Pimm et al. 1988; Forney \& Gilpin 1989; Bengtsson \& Milbrink 1995). Other studies reveal no significant relationship (Bengtsson 1989; Pollard \& Yates 1992) or provide evidence for a negative relationship (Schoener 1991; Schoener \& Spiller 1992; Lima et al. 1996; for discussions of the statistical validity of several of these studies see Diamond \& Pimm 1993; Pimm 1993; Tracy \& George1993; Gaston \& McArdle 1994).
        \item numerous theoretical treatments besides those described above (MacArthur \& Wilson 1967; MacArthur 1972; Richter-Dyn \& Goel 1972; Leigh 1975, 1981; Belovsky 1987; Goodman 1987) yield the same prediction: increased population variability leads to increased extinction risk.
    \end{itemize}
    
    
    
\todo{we need to use several measures of variability (multi-dimensional \cite{arnoldi_inherent_2019}}

\begin{itemize}
    \item definition
    \begin{itemize}
        \item collapse probability
        \item variability
            \begin{itemize}
                \item all
                \item only
                \item until
                \item time regime 10
            \end{itemize}
        \item average
        \item time rotation
    \end{itemize}
    \item limitations, issue (discussions)
    \begin{itemize}
        \item use the same simulation
        \item same fire
        \item same time
    \end{itemize}
\end{itemize}



%%%%%%%%%%%%%%%%%%%%%%%%%%%%%%%%%%%%%%%%%%%%%%%%%%%%%%%%%%%%%%%%%%%%%%%%%%%%%%%%%%%%%%%%%%%%%%%%%%%%%%%%%%%%%%%%%%
% Result
%%%%%%%%%%%%%%%%%%%%%%%%%%%%%%%%%%%%%%%%%%%%%%%%%%%%%%%%%%%%%%%%%%%%%%%%%%%%%%%%%%%%%%%%%%%%%%%%%%%%%%%%%%%%%%%%%%


\newpage
\section{Results}

\subsection{Link literature model}

\begin{itemize}
    \item link collapse probability and the parameter (enough data ?)
    \item link variability with the parameter in the literature ? ? ? 
\end{itemize}

\subsection{Loop}

\begin{itemize}
    \item show the loop
    \item illustrate different part of the loop with several figures
    \begin{itemize}
        \item no cp no var
        \item high variability low cp
        \item high variability high cp
    \end{itemize}
\end{itemize}

\subsection{Study of the peaks}

\paragraph{}
study the order and and the distance (on frequency scale) between the different peaks (variability, collapse probability and average)


\subsection{Same fire}

\paragraph{}
Instead of doing a lot of simulation and average the effect, used only one simulation (only one fire). One problem is that when we change the frequency, we need to choose between used the same time scale, and so not take the same fire (it will truncated) or used the same fire and so no have the same time scale. Because both are relevant, both are computed. 

We can remark that "same fire" tend to be more robust in the sens that it is more smooth.


%%%%%%%%%%%%%%%%%%%%%%%%%%%%%%%%%%%%%%%%%%%%%%%%%%%%%%%%%%%%%%%%%%%%%%%%%%%%%%%%%%%%%%%%%%%%%%%%%%%%%%%%%%%%%%%%%%
% Discussion
%%%%%%%%%%%%%%%%%%%%%%%%%%%%%%%%%%%%%%%%%%%%%%%%%%%%%%%%%%%%%%%%%%%%%%%%%%%%%%%%%%%%%%%%%%%%%%%%%%%%%%%%%%%%%%%%%%
\newpage
\section{Discussion}
\todo{Try to not make repetition (or not too much)}



\subsection{Model}


\paragraph{}
We assume that the frequency of the fire is independent of the density biomass. It can be argued that fuels have an influence on both severity and frequency  of fires \cite{schoennagel_interaction_2004}. However, adding this feedback (from $W$ to the frequency) will surely tends to decrease the density biomass of $W$ (because, when $W$ is higher, the frequency is higher too, and so the dynamics keep a low value of $W$). In other word, this should to have the "management fuel" scenarios more often

Also, we only consider one kind of death wood, this can be details in several type (coarse woody debris, fine woody debris, below ground ...) \cite{russell_quantifying_2015}. In the literature, the data are rarely for all the wood, and, some wood burned easier than other. 

Moreover, in practice we can distinguish several fire regime (e.g., crown fires, severe surface fires, and light surface fires) \cite{reichle_fire_1981}. All of this different fires disturbed differently the dry wood. So the dynamic can be modelled in a more complex ways.

Less relevant, we only consider density biomass. However, the spatial distribution can affect fires propagation (especially for small fires). For example, a burned area can create an obstacle when another fire occur \cite{bergeron_natural_2002}.

Even if fire is the main perturbation of the system, other disturbances can be relevant, like mechanical thinning \cite{liu_analyzing_2010}\cite{schoennagel_interaction_2004}\cite{wimberly_assessing_2009}.


\subsection{Link literature model}

\paragraph{}
Data varies greatly and also depend on several variable (type of forest, localisation ... )


\subsection{Technicality about variability}

\paragraph{}
Because the collapse of the system affect the variability of this system, it is difficult to have robust computation of the variability. None of the different variant of the variability presented above are perfect which is biased. "Variability all" compute the variance even if the system collapse, which tend to decrease a lot the variability. 
    
Also, "variability only" who compute the variability only when the system do not collapse, is biased because the variability could be not be the same when the system will collapse or not. (We do not catch the link between variability and the collapse off the system). And if the system always collapse (which can be the case, especially for long time study) we do not have an estimation of the variability. 

On the other hand, "variability until" used all the simulation, but the time of the study is never the same.

More generally, because collapse can occur at different time (or never) it is difficult to have a robust computation of the variability.



\subsection{Technicality about collapse probability}

\begin{itemize}
    \item collapse probability depend of the time of the study
    \item problem when cp is too high, to have enough data on the variability
    \item also, problem to have a measure who depend on a numerical parameter
    \item solution : used collapse probability per time units
\end{itemize}


\newpage
%%%%%%%%%%%%%%%%%%%%%%%%%%%%%%%%%%%%%%%%%%%%%%%%%%%%%%%%%%%%%%%%%%%%%%%%%%%%%%%%%%%%%%%%%%%%%%%%%%%%%%%%%%%%%%%%%%
% Conclusion
%%%%%%%%%%%%%%%%%%%%%%%%%%%%%%%%%%%%%%%%%%%%%%%%%%%%%%%%%%%%%%%%%%%%%%%%%%%%%%%%%%%%%%%%%%%%%%%%%%%%%%%%%%%%%%%%%%

\section*{Conclusion}
\addcontentsline{toc}{section}{Conclusion}


\subsection*{Synthesis}
\addcontentsline{toc}{subsection}{Synthesis}


\subsection*{Opening}
\addcontentsline{toc}{subsection}{Opening}

\paragraph{}
Talk about a more general / different problem ...


\newpage

\section*{personal review}
\addcontentsline{toc}{section}{personal review}


\newpage
\bibliographystyle{plain}
\bibliography{references}



\newpage
\appendix
\addcontentsline{toc}{section}{Annexes}

\newpage
\section{Adimensionnalisation}

\todo{calculus to Adimensionnalise the system}


\newpage
\section{Technicality}

\subsection{Solve the system}

\paragraph{}
problem : compute a continuous dynamics with discrete disturbances \\
solution : use classical solver between each fire and stop the solver when a fire occur to compute and remove the biomass


\subsection{Choice of the time step}
illustrate the problem of the choice of the time step, according to the frequency.

\subsection{}
Time of the study



\newpage
\section{Stability}

\todo{Talk about the general concept of stability (across the different disciplines)}

\end{document}
